\section{Existing Systems}

There are a variety of different rovers which have been created over the years. However, the majority of wireless robots tend to be flying drones, which obviously must be wireless. There are a few standouts for land-based drones though:

\begin{enumerate}
	\item \textbf{NASA's \textit{Curiosity} rover${}^{\cite{MARSRover}}$:}
	
		Perhaps the most famous land-based wireless rover in recent times is the \textit{Curiosity} rover, which landed on Mars on $5^{th}$ August, 2012, replacing the NASA rovers \textit{Spirit} and \textit{Opportunity} which were already on the planet. 
		
		According to a NASA missive, it's objective was to:
		
		\begin{displayquote}
			\textit{Study whether the Gale Crater area of Mars has evidence of past and present habitable environments. These studies will be part of a broader examination of past and present processes in the Martian atmosphere and on its surface. The research will use 10 instrument-based science investigations.}
		\end{displayquote}
		
		Following true to its noble mission, the Mars Rover still roams the Red Planet today. Since Mars is over 15 light-minutes away from Earth, any message or command sent over requires a quarter of an hour to reach. For a two way receive-comprehend-command missive, it takes half an hour. Thus, the Rover requires some Artificial Intelligence to traverse the dusty Martian plains on its own, and cannot be completely human-guided. Among some of the cool gadgets given to the rover are solar charging facilities and a superlaser ${}^{\ref{fig:Curiosity_Mars_Rover}}$ which can melt rock. Very cool.
		
		\begin{figure}
			\centering
			\includegraphics[width=0.7\linewidth]{"Curiosity_Mars_Rover.bmp"}
			\caption{Curiosity, the Mars Rover built by NASA. Obviously wireless.}
			\label{fig:Curiosity_Mars_Rover}
		\end{figure}
		
		
	\item \textbf{\textit{A Review of Wireless Technology Usage for Mobile Robot Controller}${}^{\cite{WirelessRobotReview}}$:} 
	
		This paper, written by Kahar et. al. provides a thorough overview of the  different wireless technology usage for mobile robot controller such as Bluetooth, WiFi or Wireless LAN and 3G. It concludes that each
		technology have the advantages and disadvantages. 
		\begin{enumerate}
			\item \textit{Bluetooth}: mobile robots can be easily handled with a push button from any electronic gadget such as mobile phone and can used to control many other
			Bluetooth enabled devices such as printers, personal computer etc. This scenario makes the mobile robot useful and marketable for real time applications, as well as cost-effective.
			
			\item \textit{IEEE 802.11 (Wifi)}: one of the advantages is reducing cost. However, a reduced performance is noticed in multi-floor and dense indoor environments because signal reflections and dynamic network conditions can result in undependable signal readings.
			
			\item \textit{3G}: used for long range supervises and control of mobile robot. This generation
			combines wireless communication and multimedia technology. In addition, 3G have high speed transmission, fast connection, and is cheap.
		\end{enumerate}
		
	
	\item \textbf{Wireless controlled robotic automation system:} 
	
		In this MTech final year project paper by Sanjeet Kumar Behera, the objective is described to be:
		
		\begin{displayquote}
			The project is to develop a controller and to control a 6 DOF robotic arm for Pick \& Place Application over wireless. The objective is to learn various types of control methods for the pick and place robotic arm for educational purpose uses. The 6-DOF robot arm is controlled by a serial servo controller circuit board. The controller board utilizes a Atmega328 microcontroller ( Boot loaded with Arduino Diecimila Bootloader ) from Atmel Corporation as the control system to control all the activities. 
			
			The input sensors like potentiometers will send a the input signals to the microcontroller, then microcontroller will analyze the data accordingly and will send control signals to the output devices. This output signal basically turns ON or OFF the output devices such as servo motors. The servo controller board is connected to the serial port on a PC running the Microsoft Windows operating system. The ATMega328 will be programmed to run robot arm sequences independently by help of a FT232RL breakout board. Arduino integrated development environment (IDE) is used to develop the arduino sketch.
			
			
		\end{displayquote}
		
		\begin{figure}
			\centering
			\includegraphics[width=0.7\linewidth]{WirelessNITRoorkie}
			\caption{6DOF Robotic arm to pick and move objects.}
			\label{fig:WirelessNITRoorkie}
		\end{figure}
		
		
		
	 


\end{enumerate}