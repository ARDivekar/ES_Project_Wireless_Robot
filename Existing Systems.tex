\section{Existing Systems}

There are a variety of different rovers which have been created over the years. However, the majority of wireless robots tend to be flying drones, which obviously must be wireless. There are a few standouts for land-based drones though:

\begin{enumerate}
	\item \textbf{NASA's \textit{Curiosity} rover${}^{\cite{MARSRover}}$:}
	
		Perhaps the most famous land-based wireless rover in recent times is the \textit{Curiosity} rover, which landed on Mars on $5^{th}$ August, 2012, replacing the NASA rovers \textit{Spirit} and \textit{Opportunity} which were already on the planet. 
		
		According to a NASA missive, it's objective was to:
		
		\begin{displayquote}
			\textit{Study whether the Gale Crater area of Mars has evidence of past and present habitable environments. These studies will be part of a broader examination of past and present processes in the Martian atmosphere and on its surface. The research will use 10 instrument-based science investigations.}
		\end{displayquote}
		
		Following true to its noble mission, the Mars Rover still roams the Red Planet today. Since Mars is over 15 light-minutes away from Earth, any message or command sent over requires a quarter of an hour to reach. For a two way receive-comprehend-command missive, it takes half an hour. Thus, the Rover requires some Artificial Intelligence to traverse the dusty Martian plains on its own, and cannot be completely human-guided. Among some of the cool gadgets given to the rover are solar charging facilities and a superlaser ${}^{\ref{fig:Curiosity_Mars_Rover}}$ which can melt rock. Very cool.
		
		\begin{figure}
			\centering
			\includegraphics[width=0.7\linewidth]{"Curiosity_Mars_Rover.bmp"}
			\caption{Curiosity, the Mars Rover built by NASA. Obviously wireless.}
			\label{fig:Curiosity_Mars_Rover}
		\end{figure}


\end{enumerate}